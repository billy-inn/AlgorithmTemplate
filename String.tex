% String.tex
\chapter{String}
 \section{KMP}
 \begin{lstlisting}[language=C++]
//fail function
int f[maxn];
void getfail(char* P)
{
    int m=strlen(P);
    f[0]=0;f[1]=0;
    for(int i=1;i<m;i++)
    {
        int j=f[i];
        while(j && P[i]!=P[j]) j=f[j];
        f[i+1]=P[i]==P[j]?j+1:0;
    }
}

//find function
void find(char* T,char* P)
{
    int n=strlen(T),m=strlen(P);
    int j=0;
    for(int i=0;i<n;i++)
    {
        while(j && P[j]!=T[i]) j=f[j];
        if(P[j]==T[i]) j++;
        if(j==m)
        {
            operate();
            j=f[j];     //note when find a match, remember to get backward along fail edge once
        }
    }
}
 \end{lstlisting}

 \section{AC Automation}
 \begin{lstlisting}[language=C++]
//tire tree
const int maxn=100010;//maximal node number
const int sigma=26;   
//the size of character size,the value of characters are 0~sigma-1
int ch[maxn][sigma];  //ch[i][j] means the node going from i along j to
int val[maxn];        //additive information
int sz;               //the total number of node
void init()           //initial
{
    sz=1;
    memset(ch[0],0,sizeof(ch[0]));
}
int idx(char c)       //index the character
{
    return c-'a';     //vary depending on specific condition
}
//insert string s,whose additive information is v
//note that v cannot be 0
void insert(char *s,int v)
{
    int u=0,n=strlen(s);
    for(int i=0;i<n;i++)
    {
        int c=idx(s[i]);
        if(!ch[u][c])
        {
            memset(ch[sz],0,sizeof(ch[sz]));
            val[sz]=0;
            ch[u][c]=sz++;
        }
        u=ch[u][c];
    }
    val[u]=v;
    //val[u]|=(1<<v);
    //operating like this when it comes to state compressing
}

//get fail function
int f[maxn];            //fail function
int last[maxn];         //suffix link
void getfail()
{
    queue<int> q;
    f[0]=0;
    for(int c=0;c<sigma;c++)
    {
        int u=ch[0][c];
        if(u) {f[u]=0;q.push(u);last[u]=0;}
    }
    while(!q.empty())
    {
        int r=q.front();q.pop();
        for(int c=0;c<sigma;c++)
        {
            int u=ch[r][c];
            if(!u)
            {
                //ch[r][c]=ch[f[r]][c];
                //add all nonexist edges, applying on dp
                continue;
            }
            q.push(u);
            int v=f[r];
            while(v && !ch[v][c]) v=f[v];//when all nonexist edges are added, code here can be deleted
            f[u]=ch[v][c];
            last[u]=val[f[u]]?f[u]:last[f[u]];
        }
    }
}

int find(char *T)
{
    int n=strlen(T);
    int j=0;
    for(int i=0;i<n;i++)
    {
        int c=idx(T[i]);
        while(j && !ch[j][c]) j=f[j];   //when all nonexist edges are added, code here can be deleted
        j=ch[j][c];
        if(val[j]) process(j);
        else if(last[j]) process(last[j]);
    }
}

void process(int j)
{
    if(j)
    {
        operate();      //vary depending on specific condition
        process(last[j]);
    }
}
 \end{lstlisting}

 \section{Suffix Array}
 \begin{lstlisting}[language=C++]
int s[maxn];                   //string to be constructed
int sa[maxn];                  //suffix array
int t[maxn],t2[maxn],c[maxn];
//every character's value is 0-m-1,n is usually the length of string plus 1
 void build_sa(int m,int n)           //construct suffix array
{
    int i,*x=t,*y=t2;
    for(i=0;i<m;i++) c[i]=0;
    for(i=0;i<n;i++) c[x[i]=s[i]]++;
    for(i=1;i<m;i++) c[i]+=c[i-1];
    for(i=n-1;i>=0;i--) sa[--c[x[i]]]=i;
    for(int k=1;k<=n;k<<=1)
    {
        int p=0;
        for(i=n-k;i<n;i++) y[p++]=i;
        for(i=0;i<n;i++) if(sa[i]>=k) y[p++]=sa[i]-k;
        for(i=0;i<m;i++) c[i]=0;
        for(i=0;i<n;i++) c[x[y[i]]]++;
        for(i=0;i<m;i++) c[i]+=c[i-1];
        for(i=n-1;i>=0;i--) sa[--c[x[y[i]]]]=y[i];
        swap(x,y);
        p=1;x[sa[0]]=0;
        for(i=1;i<n;i++)
            x[sa[i]]=y[sa[i-1]]==y[sa[i]]&&y[sa[i-1]+k]==y[sa[i]+k]?p-1:p++;
        if(p>=n) break;
        m=p;
    }
}

//calculate rank and height array
int rank[maxn];         //suffix i's index in sa
int height[maxn];       //the LCP of sa[i-1] and sa[i]
void getheight(int n) //n is the length of string
{
    int i,j,k=0;
    for(i=0;i<=n;i++) rank[sa[i]]=i;
    for(i=0;i<n;i++)
    {
        if(k) k--;
        int j=sa[rank[i]-1];
        while(s[i+k]==s[j+k]) k++;
        height[rank[i]]=k;
    }
}

//RMQ
int d[maxn][50];
void RMQ_init(int n)
{
    for(int i=0;i<n;i++) d[i][0]=height[i];
    for(int j=1;(1<<j)<=n;j++)
        for(int i=0;i+(1<<j)-1<n;i++)
            d[i][j]=min(d[i][j-1],d[i+(1<<(j-1))][j-1]);
}
int RMQ(int L,int R)
{
    if(L>R) swap(L,R);
    if(L==R) return L-sa[L];
    L++;
    int k=0;
    while((1<<(k+1))<=R-L+1) k++;
    return min(d[L][k],d[R-(1<<k)+1][k]);
}

//inital
build_sa(m,n+1);
getheight(n);
RMQ_init(n+1);
 \end{lstlisting}

 \section{Suffix Automation}
 \begin{lstlisting}[language=C++]
char s[maxn];               //string to be constructed
int sz;                     //the number of node
struct state
{  
    state *pre;
    state *go[sigma];
    int val;                //the length of the longest path
    void clear()            //initialize node
    {  
        pre=0;val=0;  
        memset(go,0,sizeof(go));  
    }  
};  
state *root,*last;          //root node and last node added
state st[maxn];  
void init()                 //initialize SAM
{  
    sz=0;  
    root=last=&st[sz++];  
    root->clear();  
}  
void extend(int w)          //extend node
{  
    state *p=last;          //last node added
    state *np=&st[sz++];    //create a new node
    np->clear();
    np->val=p->val+1;  
    while(p && p->go[w]==0) //add edge to np when p->go[w]=0
    {  
        p->go[w]=np;  
        p=p->pre;  
    }  
    if(p==0) np->pre=root;  //if w occurs first,add edge to the root
    else  
    {  
        state *q=p->go[w];  
        if(q->val==p->val+1)    //q is the proper state
            np->pre=q;  
        else  
        {  
            state *nq=&st[sz++];//create a new node and copy the information
            nq->clear();  
            nq->val=p->val+1;  
            memcpy(nq->go,q->go,sizeof(q->go));  
            nq->pre=q->pre;  
            q->pre=nq;
            np->pre=nq;  
            while(p && p->go[w]==q)
            {  
                p->go[w]=nq;  
                p=p->pre;  
            }  
        }  
    }  
    last=np;  
}

//sort the nodes
int t[maxn];
state *r[maxn];
//l is the length of string,sz is the number of node
//note here contains root node,root node is indexed as 1 in r
for(int i=0;i<l;i++) t[i]=0;
for(int i=0;i<sz;i++) t[st[i].val]++;
for(int i=1;i<l;i++) t[i]+=t[i-1];
for(int i=0;i<sz;i++) r[t[st[i].val]--]=&st[i];
 \end{lstlisting}

 \section{String Hash}
 \begin{lstlisting}[language=C++]
typedef unsigned long long ULL;
const int maxn=10000;
const int x=123;        //hash's parameter
ULL H[maxn];            //the hash value of string
ULL xp[maxn];           //x^n
void hash_init(char *s) //initialize H(n),x^n
{
    int n=strlen(s);
    H[n]=0;
    for(int i=n-1;i>=0;i--) H[i]=H[i+1]*x+(s[i]-'a');   //calculate H(n)
    xp[0]=1;
    for(int i=1;i<=n;i++) xp[i]=xp[i-1]*x;              //calculate x^n
}
//hash function,return the hash value of string starts from i with length of L
ULL hash(int i,int L)
{
    return H[i]-H[i+L]*xp[L];
}
 \end{lstlisting}

 \section{Longest Palindromic Substring}
 \textbf{Test:LeetCode, Longest Palindromic Substring}
 \begin{lstlisting}[language=C++]
int p[2010];
void get(string str)
{
	int mx=0,id;
	for(int i=1;i<str.size();i++)
	{
		if(mx>i) p[i]=min(p[2*id-i],mx-i);
		else p[i]=1;
		for(;str[i+p[i]]==str[i-p[i]];p[i]++)
			;
		if(p[i]+i>mx)
		{
			mx=p[i]+i;
			id=i;
		}
	}
}
string longestPalindrome(string s) 
{
	string t="$";
	for(int i=0;i<s.size();i++)
		t+='#',t+=s[i];
	t+='#';
	get(t);
	int ans=0,pos;
	string str="";
	for(int i=1;i<t.size();i++)
		if(p[i]>ans)
		{
			ans=p[i];
			pos=i;
		}
	for(int i=pos-ans+1;i<pos+ans;i++)
		if(t[i]!='#') str+=t[i];
	return str;
}
 \end{lstlisting}



















\endinput