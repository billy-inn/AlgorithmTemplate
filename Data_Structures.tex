% Data_Structures.tex
\chapter{Data Structures}
 \section{Fenwick Tree}
  \textbf{Test: CF 383C}
  \subsection{unidimension}
  \begin{lstlisting}[language=C++]
int C[maxn];
int lowbit(int x){return x&-x;}
void add(int x,int v) 
{
    while(x<=n)
    {
        C[x]+=v;x+=lowbit(x);
    }
}
int getsum(int x)
{
    int ans=0;
    while(x>0)
    {
        ans+=C[x];x-=lowbit(x);
    }
    return ans;
}
  \end{lstlisting}
  \subsection{two-dimension}
  \begin{lstlisting}[language=C++]
int C[maxn][maxn];
int n,m;
int lowbit(int x){return x&-x;}
void add(int x,int y,int v)
{
    for(int i=x;i<=n;i+=lowbit(i))
        for(int j=y;j<=m;j+=lowbit(j))
        {
            C[i][j]+=v;
        }
}
int getsum(int x,int y) 
{
    int ans=0;
    for(int i=x;i>0;i-=lowbit(i))
        for(int j=y;j>0;j-=lowbit(j))
        {
            ans+=C[i][j];
        }
    return ans;
}
  \end{lstlisting}

 \section{RMQ}
 \textbf{Test: CF 6E}
 \begin{lstlisting}[language=C++]
int d[maxn][power];
void RMQ_init(int* A,int n)
{
	for(int i=0;i<n;i++) d[i][0]=A[i];
	for(int j=1;(1<<j)<=n;j++)
		for(int i=0;i+(1<<j)-1<n;i++)
			d[i][j]=min(d[i][j-1],d[i+(1<<(j-1))][j-1]);

}
int RMQ(int L,int R)
{
	int k=0;
	while((1<<(k+1))<=R-L+1) k++;
	return min(d[L][k],d[R-(1<<k)+1][k]);
}
 \end{lstlisting}

 \section{Splay Tree}
  \subsection{Implemented by List}
  \textbf{Test: hdu1754,1890,3436,3487,poj3468,UVa11922}
  \begin{lstlisting}[language=C++]
const int maxn=200010;
struct Node
{
    Node *ch[2];    //children
    Node *pre;      //ancestor
    int sz;         //the number of nodes that are rooted in this node
    int val;        //the value to be maintained
    inline void push_up()       //like the function of segment tree
    {
        sz=1+ch[0]->sz+ch[1]->sz;
    }
    inline void push_down()     //like the function of segment tree
};
Node *null=new Node();          //virtual node
struct SplaySequence
{
    Node seq[maxn];
    Node *root;
    int a[maxn];
    Node* build(int l,int r,Node *f)
    {
        if(l>r) return null;
        int m=(l+r)>>1;
        Node *o=&seq[m];
        o->pre=f;
        o->val=a[m];
        o->ch[0]=build(l,m-1,o);
        o->ch[1]=build(m+1,r,o);
        o->push_up();
        return o;
    }
    void init(int sz)
    {
        null->sz=0;
        root=build(1,sz,null);
    }
    //d=0 rotate toward the left、d=1 rotate toward the right
    inline void rotate(Node *x,int d)
    {
        Node *y=x->pre;
        y->push_down();
        x->push_down();
        y->ch[d^1]=x->ch[d];
        if(x->ch[d]!=null) x->ch[d]->pre=y;
        x->pre=y->pre;
        if(y->pre!=null) y->pre->ch[y->pre->ch[1]==y]=x;
        x->ch[d]=y;
        y->pre=x;
        y->push_up();
        if(y==root) root=x;
    }
    //rotate x to the position below f and return x
    inline Node* splay(Node *x,Node *f)
    {
        while(x->pre!=f)
        {
            if(x->pre->pre==f) rotate(x,x->pre->ch[0]==x);
            else
            {
                Node *y=x->pre,*z=y->pre;
                int d=(z->ch[0]==y);
                if(y->ch[d]==x) rotate(x,d^1),rotate(x,d);
                else rotate(y,d),rotate(x,d);
            }
        }
        x->push_up();
        return x;
    }
    //rotate the node that is rooted in r 
    //and ranks k to the position below f, and return it
    inline Node* select(int k,Node *f,Node *r)
    {
        Node *t=r;
        while(1)
        {
            int s=t->r-t->l+1;
            int tmp=t->ch[0]->sz;
            if(k<=tmp+s && k>tmp) break;
            if(k<=tmp) t=t->ch[0];
            else k-=tmp+s,t=t->ch[1];
        }
        return splay(t,f);
    }
    //split o to two trees
    //the first k nodes are rooted in l, the other are rooted in r
    inline void split(Node *o,int k,Node* &l,Node* &r)
    {
        l=select(k,null,o);
        r=l->ch[1];
        l->ch[1]=null;
        r->pre=null;
        l->push_up();
    }
    //merge l and r, and return the root
    inline Node* merge(Node *l,Node *r)
    {
        Node *t=select(l->sz,null,l);
        t->ch[1]=r;
        r->pre=t;
        t->push_up();
        return t;
    }
} ss;
void debug(Node *o)     //debug function
{
    if(o!=null)
    {
        debug(o->ch[0]);
        printf("%d\n",o->val);
        debug(o->ch[1]);
    }
}
  \end{lstlisting}
  \subsection{Implemented by array}
  \textbf{Test: POJ 3468}
  \begin{lstlisting}[language=C++]
#include <cstdio>
#define keyTree (ch[ ch[root][1] ][0])
const int maxn = 222222;
struct SplayTree{
	int sz[maxn];
	int ch[maxn][2];
	int pre[maxn];
	int root , top1 , top2;
	int ss[maxn] , que[maxn];
 
	inline void Rotate(int x,int f) {
		int y = pre[x];
		push_down(y);
		push_down(x);
		ch[y][!f] = ch[x][f];
		pre[ ch[x][f] ] = y;
		pre[x] = pre[y];
		if(pre[x]) ch[ pre[y] ][ ch[pre[y]][1] == y ] = x;
		ch[x][f] = y;
		pre[y] = x;
		push_up(y);
	}
	inline void Splay(int x,int goal) {
		push_down(x);
		while(pre[x] != goal) {
			if(pre[pre[x]] == goal) {
				Rotate(x , ch[pre[x]][0] == x);
			} else {
				int y = pre[x] , z = pre[y];
				int f = (ch[z][0] == y);
				if(ch[y][f] == x) {
					Rotate(x , !f) , Rotate(x , f);
				} else {
					Rotate(y , f) , Rotate(x , f);
				}
			}
		}
		push_up(x);
		if(goal == 0) root = x;
	}
	inline void RotateTo(int k,int goal) {
		int x = root;
		push_down(x);
		while(sz[ ch[x][0] ] != k) {
			if(k < sz[ ch[x][0] ]) {
				x = ch[x][0];
			} else {
				k -= (sz[ ch[x][0] ] + 1);
				x = ch[x][1];
			}
			push_down(x);
		}
		Splay(x,goal);
	}
	inline void erase(int x) {
		int father = pre[x];
		int head = 0 , tail = 0;
		for (que[tail++] = x ; head < tail ; head ++) {
			ss[top2 ++] = que[head];
			if(ch[ que[head] ][0]) que[tail++] = ch[ que[head] ][0];
			if(ch[ que[head] ][1]) que[tail++] = ch[ que[head] ][1];
		}
		ch[ father ][ ch[father][1] == x ] = 0;
		pushup(father);
	}
//above usually  unchanged//////////////////////////////////////////////

	inline void NewNode(int &x,int c) {
		if (top2) x = ss[--top2];
		else x = ++top1;
		ch[x][0] = ch[x][1] = pre[x] = 0;
		sz[x] = 1;
 
		val[x] = sum[x] = c;/*specialized function*/
		add[x] = 0;
	}
 
	inline void push_down(int x) {/*specialized function*/
		if(add[x]) {
			val[x] += add[x];
			add[ ch[x][0] ] += add[x];
			add[ ch[x][1] ] += add[x];
			sum[ ch[x][0] ] += (long long)sz[ ch[x][0] ] * add[x];
			sum[ ch[x][1] ] += (long long)sz[ ch[x][1] ] * add[x];
			add[x] = 0;
		}
	}
	inline void push_up(int x) {
		sz[x] = 1 + sz[ ch[x][0] ] + sz[ ch[x][1] ];
		/*specialized function*/
		sum[x] = add[x] + val[x] + sum[ ch[x][0] ] + sum[ ch[x][1] ];
	}
 
	/*initialize*/
	inline void makeTree(int &x,int l,int r,int f) {
		if(l > r) return ;
		int m = (l + r)>>1;
		NewNode(x , num[m]);		/*varies according to the problem*/
		makeTree(ch[x][0] , l , m - 1 , x);
		makeTree(ch[x][1] , m + 1 , r , x);
		pre[x] = f;
		push_up(x);
	}
	inline void init(int n) {/*specialized funciton*/
		ch[0][0] = ch[0][1] = pre[0] = sz[0] = 0;
		add[0] = sum[0] = 0;
 
		root = top1 = 0;
		//for convience to add two nodes
		NewNode(root , -1);
		NewNode(ch[root][1] , -1);
		pre[top1] = root;
		sz[root] = 2;
 
 
		for (int i = 0 ; i < n ; i ++) scanf("%d",&num[i]);
		makeTree(keyTree , 0 , n-1 , ch[root][1]);
		push_up(ch[root][1]);
		push_up(root);
	}
	inline void update( ) {/*specialized funciton*/
		int l , r , c;
		scanf("%d%d%d",&l,&r,&c);
		RotateTo(l-1,0);
		RotateTo(r+1,root);
		add[ keyTree ] += c;
		sum[ keyTree ] += (long long)c * sz[ keyTree ];
	}
	inline void query() {/*specialized function*/
		int l , r;
		scanf("%d%d",&l,&r);
		RotateTo(l-1 , 0);
		RotateTo(r+1 , root);
		printf("%lld\n",sum[keyTree]);
	}
 
	int num[maxn];
	int val[maxn];
	int add[maxn];
	long long sum[maxn];
}spt;
 
 
int main() {
	int n , m;
	scanf("%d%d",&n,&m);
	spt.init(n);
	while(m --) {
		char op[2];
		scanf("%s",op);
		if(op[0] == 'Q') {
			spt.query();
		} else {
			spt.update();
		}
	}
	return 0;
}
  \end{lstlisting}










\endinput