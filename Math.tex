% Math.tex
\chapter{Math}
 \section{High Precision Class}
  \begin{lstlisting}[language=C++]
const int maxn = 200;
struct bign{
  int len, s[maxn];
  bign() {
    memset(s, 0, sizeof(s));
    len = 1;
  }
  bign(int num) {
    *this = num;
  }
  bign(const char* num) {
    *this = num;
  }
  bign operator = (int num) {
    char s[maxn];
    sprintf(s, "%d", num);
    *this = s;
    return *this;
  }
  bign operator = (const char* num) {
    len = strlen(num);
    for(int i = 0; i < len; i++) s[i] = num[len-i-1] - '0';
    return *this;
  }
  string str() const {
    string res = "";
    for(int i = 0; i < len; i++) res = (char)(s[i] + '0') + res;
    if(res == "") res = "0";
    return res;
  }
  bign operator + (const bign& b) const{
    bign c;
    c.len = 0;
    for(int i = 0, g = 0; g || i < max(len, b.len); i++) {
      int x = g;
      if(i < len) x += s[i];
      if(i < b.len) x += b.s[i];
      c.s[c.len++] = x % 10;
      g = x / 10;
    }
    return c;
  }
  void clean() {
    while(len > 1 && !s[len-1]) len--;
  }
  bign operator * (const bign& b) {
    bign c; c.len = len + b.len;
    for(int i = 0; i < len; i++)
      for(int j = 0; j < b.len; j++)
        c.s[i+j] += s[i] * b.s[j];
    for(int i = 0; i < c.len-1; i++){
      c.s[i+1] += c.s[i] / 10;
      c.s[i] %= 10;
    }
    c.clean();
    return c;
  }
  bign operator - (const bign& b) {
    bign c; c.len = 0;
    for(int i = 0, g = 0; i < len; i++) {
      int x = s[i] - g;
      if(i < b.len) x -= b.s[i];
      if(x >= 0) g = 0;
      else {
        g = 1;
        x += 10;
      }
      c.s[c.len++] = x;
    }
    c.clean();
    return c;
  }
  bool operator < (const bign& b) const{
    if(len != b.len) return len < b.len;
    for(int i = len-1; i >= 0; i--)
      if(s[i] != b.s[i]) return s[i] < b.s[i];
    return false;
  }
  bool operator > (const bign& b) const{
    return b < *this;
  }
  bool operator <= (const bign& b) {
    return !(b > *this);
  }
  bool operator == (const bign& b) {
    return !(b < *this) && !(*this < b);
  }
  bign operator += (const bign& b) {
    *this = *this + b;
    return *this;
  }
};
istream& operator >> (istream &in, bign& x) {
  string s;
  in >> s;
  x = s.c_str();
  return in;
}
ostream& operator << (ostream &out, const bign& x) {
  out << x.str();
  return out;
}
  \end{lstlisting}

 \section{Number Theory}
  \subsection{Sieve of Eratosthenes}
  \begin{lstlisting}[language=C++]
const int maxn=10000010;
const int maxp=700000;
int vis[maxn],prime[maxp];
void sieve(int n)	//sieve the primes no larger than n
{
	int m=(int)sqrt(n+0..5);
	memset(vis,0,sizeof(vis));
	for(int i=2;i<=m;i++) if(!vis[i])
		for(int j=i*i;j<=n;j+=i) vis[j]=1;
	int c=0;
	for(int i=2;i<=n;i++) if(!vis[i])
		prime[c++]=i;
}
  \end{lstlisting}
  \subsection{Sieve of Euler}
\textbf{The Sieve can be used to solve problem of multiplicative function.\\
Test: CF 10C}
  \begin{lstlisting}[language=C++]
int vis[maxn],prime[maxp];
void sieve(int n)
{
	int tot=0;
	for(int i=2;i<=n;i++)
	{
		if(!vis[i]) prime[tot++]=i;
		for(int j=0;j<tot;j++)
		{
			if(i*prime[j]>n) break;
			vis[i*prime[j]]=1;
			if(i%prime[j]==0) break;
		}
}
  \end{lstlisting}
  \subsection{Euclid Algorithm}
  \begin{lstlisting}[language=C++]
typedef long long LL;
LL gcd(LL a,LL b)
{
	return b==0?a:gcd(b,a%b);
}
void exgcd(LL a,LL b,LL& d,LL& x,LL& y)
{
	if(!b){d=a;x=1;y=0;}
	else
	{
		gcd(b,a%b,d,y,x);
		y-=x*(a/b);
	}
}
  \end{lstlisting}
  \subsection{Euler Function}
  \begin{lstlisting}[language=C++]
int euler_phi(int n)
{
	int m=(int)sqrt(n+0.5);
	int ans=n;
	for(int i=2;i<=m;i++) if(n%i==0)
	{
		ans=ans/i*(i-1);
		while(n%i==0) n/=i;
	}
	if(n>1) ans=ans/n*(n-1);
}
int phi[maxn];
void phi_table(int n)
{
	for(int i=2;i<=n;i++) phi[i]=0;
	phi[1]=1;
	for(int i=2;i<=n;i++) if(!phi[i])
	{
		for(int j=i;j<=n;j+=i)
		{
			if(!phi[j]) phi[j]=j;
			phi[j]=phi[j]/i*(i-1);
		}
	}
}
  \end{lstlisting}
  \subsection{Calculate the Inversion}
  \begin{lstlisting}[language=C++]
LL inv(LL a, LL n)
{
	LL d,x,y;
	exgcd(a,n,d,x,y);
	return d==1?(x+n)%n:-1;
}
  \end{lstlisting}
  \subsection{Chinese Remainder Theorem}
  \begin{lstlisting}[language=C++]
//n equations:x=a[i](mod m[i]) (0<=i<n)
LL china(int n,int *a,int *m)
{
	LL M=1,d,y,x=0;
	for(int i=0;i<n;i++) M*=m[i];
	for(int i=0;i<n;i++)
	{
		LL w=M/m[i];
		exgcd(m[i],w,d,d,y);
		x=(x+y*w*a[i])%M;
	}
	return (x+M)%M;
}
  \end{lstlisting}
  \subsection{Bezout Theorem}
For any two integers $a$,$b$, assume that $d$ is their gcd. The equation $ax+by=m$ has integer solutions $(x,y)$ if and only if m is the multiple of d. When the equation has solutions, the set of solutions are $\{(\frac{mx_0+kb}{d},\frac{my_0-ka}{d})|k \in \mathbb{Z} \}$, where $(x_0,y_0)$ is one solution of this equation. 
 \section{Combinatorics}
  \subsection{Calculate Combinations by Recurrence}
  \begin{lstlisting}[language=C++]
void init()
{
    memset(C,0,sizeof(C));
    C[0][0]=1;
    for(int i=1;i<n;i++)
    {
        C[i][0]=C[i][i]=0;
        for(int j=1;j<i;j++)
            C[i][j]=C[i-1][j]+C[i-1][j-1];
    }
}
  \end{lstlisting}
 \section{Matrix and System of Linear Equations}
  \subsection{Matrix Fast Power}
  \begin{lstlisting}[language=C++]
typedef matrix int[maxn][maxn];
void mat_mul(matrix A,matrix B,matrix res)
{
    matrix C;
    memset(C,0,sizeof(C));
    for(int i=0;i<n;i++)
        for(int j=0;j<n;j++)
            for(int k=0;k<n;k++)
                C[i][j]+=A[i][k]*B[k][j];
    memcpy(res,C,sizeof(C));
}
void mat_pow(matrix A,int m,matrix res)
{
    matrix a,r;
    memcpy(a,A,sizeof(a));
    memset(r,0,sizeof(r));
    for(int i=0;i<n;i++) r[i][i]=1;
    while(m)
    {
        if(m&1) mat_mul(r,a,r);
        m>>=1;
        mat_mul(a,a,a);
    }
    memcpy(res,r,sizeof(r));
}
  \end{lstlisting}
 \section{Numerical Method}
  \subsection{Adaptive Simpson's Rule}
  \begin{lstlisting}[language=C++]
//F is a global function
double simpson(double a,double b)
{
	double c=a+(b-a)/2;
	return (F(a)+4*F(c)+F(b))*(b-a)/6;
}
//recursive part
double asr(double a,double b,double eps,double A)
{
	double c=a+(b-a)/2;
	double L=simpson(a,c),R=simpson(c,b);
	if(fabs(L+R-A)<=15*eps) return L+R+(L+R-A)/15.0;
	return asr(a,c,eps/2,L)+asr(c,b,eps/2,R);
}
//main part
double asr(double a,double b,double eps)
{
	return asr(a,b,eps,simpson(a,b));
}
  \end{lstlisting}


\endinput