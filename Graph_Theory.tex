% Graph_Theory.tex
\chapter{Graph Theory}
 \section{Fundamentals} 
  \subsection{Topo Sort}
  \textbf{Test:UVaLive 4255}
  \begin{lstlisting}[language=C++]
const int maxn=20;
int g[maxn][maxn];  //adjacency matrix
int in[maxn];       //indegree
int topo[maxn];     //result
int toposort(int n)
{
    int top=-1;
    for(int i=0;i<n;i++)
        if(in[i]==0)
        {
            in[i]=top;top=i;
        }
    for(int i=0;i<n;i++)
        if(top==-1) return 0;   //exist cycle
        else
        {
            int j=top;
            top=in[j];
            topo[i]=j;
            for(int k=0;k<n;k++)
                if(g[j][k] && (--in[k])==0)
                {
                    in[k]==top;
                    top=k;
                }
        }
    return 1;       //succeed
}
  \end{lstlisting}

  \subsection{Eulerian trail and circuit}
  \textbf{Test:UVa 10054,10441}
  \begin{itemize}
   \item undirected graph
   \begin{enumerate}
    \item whether there exists an Eulerian circuit
    \begin{enumerate}
	\item the graph is connected
	\item every vertex has even degree
    \end{enumerate}
    \item whether there exists an Eulerian trail
    \begin{enumerate}
	\item the graph is connected
	\item there are only two vertex with odd degree which are the start point and the end point
    \end{enumerate}
   \end{enumerate}
   \item directed graph
   \begin{enumerate}
    \item whether there exists an Eulerian circuit
    \begin{enumerate}
	\item the graph is connected
	\item every vertex has equal indegree and outdegree
    \end{enumerate}
    \item whether there exists an Eulerian trail
    \begin{enumerate}
	\item the graph is connected
	\item there is only one vertex with outdegree greater than indegree by one as the start point, and there is only one vertex with indegree greater than outdegree by one as end point
    \end{enumerate}
   \end{enumerate}
  \end{itemize}
  \begin{lstlisting}[language=C++]
int g[maxn][maxn];      //adjacency matrix
int cnt[maxn];          //degree
vector<pair<int,int> > ans;     //edge list, stored in reverse order
int n,m;                        //n is the number of vertices, m is the number of edges
void euler(int u)
{
    for(int i=0;i<n;i++) if(g[u][i])
    {
        g[u][i]--;g[i][u]--;
        euler(i);
        ans.push_back(make_pair(u,i));
    }
}
int solve(int start)
{
    int flag=1;
    for(int i=0;i<n;i++)
        if(cnt[i]%2)
        {
            flag=0;break;
        }
    if(flag)
    {
        ans.clear();
        euler(start);
        if(ans.size()!=m || ans[0].second!=ans[ans.size()-1].first) flag=0;
    }
    return flag;        //whether exists an Eulerian circuit
}
  \end{lstlisting}




 \section{Shortest Paths}
 



 \section{Minimal Spanning Tree}
  \subsection{Fundamentals}
  \begin{thm}
For an $n$-vertex graph $G$(with $n \ge 1$),the following are equivalent(and character the trees with $n$ vertices).
   \begin{enumerate}
    \item $G$ is connected and has no cycles
    \item $G$ is connected and has $n-1$ edges
    \item $G$ has $n-1$ edges and no cycles
    \item $G$ has no loops and has, for each $u,v \in V(G)$, exactly one u,v-path
   \end{enumerate}
  \end{thm}

  \begin{cor}
   \begin{enumerate}
    \item Every edge of a tree is a cut-edge
    \item Adding one edge to a tree forms exactly one cycle
    \item Every connected graph contains a spanning tree
   \end{enumerate}
  \end{cor}
  \subsection{Kruskal algorithm}
  \begin{lstlisting}[language=C++]
int cmp(const int i,const int j){return w[i]<w[j];} //indirect sorting function
int find(int x){return p[x]==x?x:p[x]=find(p[x]);}  //disjoint set
int Kruskal()
{
	int ans=0;
	for(int i=0;i<n;i++) p[i]=i; //initial disjoint set
	for(int i=0;i<m;i++) r[i]=i; //initial edge index
	sort(r,r+m,cmp); //sort edges
	for(int i=0;i<m;i++)
	{
		int e=r[i];int x=find(u[e]);int y=find(v[e]);
		if(x!=y) ans+=w[e],p[x]=y;
	}
	return ans;
}
  \end{lstlisting}
  \subsection{Second-best Minimum Spanning Tree}
  \textbf{Test:UVa 10600}
  \begin{lstlisting}[language=C++]
#include <iostream>
#include <cstdio>
#include <cstring>
#include <algorithm>
using namespace std;
const int maxn=110;
const int maxm=maxn*maxn/2;
struct edges
{
	int u,v,c;
	bool operator<(const edges& tmp) const
	{
		return c<tmp.c;
	}
} edge[maxm];
int n,m,e;
int head[maxn],pnt[maxn*2],nxt[maxn*2],cost[maxn*2];
int maxcost[maxn][maxn];
int f[maxn];
int vis[maxn],vis2[maxm];
void addedge(int u,int v,int c)
{
	pnt[e]=v;cost[e]=c;nxt[e]=head[u];head[u]=e++;
}
int find(int u){return f[u]==u?u:f[u]=find(f[u]);}
void dfs(int u,int fa,int c)
{
	maxcost[u][fa]=maxcost[fa][u]=c;
	for(int i=1;i<=n;i++) if(vis[i]) maxcost[u][i]=maxcost[i][u]=max(maxcost[i][fa],c);
	vis[u]=1;
	for(int i=head[u];i!=-1;i=nxt[i])
	{
		int v=pnt[i];
		if(v!=fa) dfs(v,u,cost[i]);
	}
}
int main()
{
	//freopen("in.txt","r",stdin);
	int T;
	scanf("%d",&T);
	while(T--)
	{
		scanf("%d%d",&n,&m);
		for(int i=0;i<m;i++) scanf("%d%d%d",&edge[i].u,&edge[i].v,&edge[i].c);
		for(int i=1;i<=n;i++) f[i]=i;
		memset(head,-1,sizeof(head));
		e=0;
		sort(edge,edge+m);
		int ans1=0;
		memset(vis2,0,sizeof(vis2));
		for(int i=0;i<m;i++)
		{
			int u=find(edge[i].u),v=find(edge[i].v);
			if(u!=v)
			{
				f[u]=v;
				addedge(u,v,edge[i].c);
				addedge(v,u,edge[i].c);
				ans1+=edge[i].c;
				vis2[i]=1;
			}
		}
		memset(vis,0,sizeof(vis));
		dfs(1,0,0);
		int ans2=0x3fffffff;
		for(int i=0;i<m;i++) if(!vis2[i])
		{
			int u=edge[i].u,v=edge[i].v,c=edge[i].c;
			ans2=min(ans2,ans1+c-maxcost[u][v]);
		}
		printf("%d %d\n",ans1,ans2);
	}
	return 0;
}
  \end{lstlisting}
  \subsection{Minimal Bottleneck Spanning Tree}
Actually, the MBST is the MST. And we can get it by Kruskal Algorithm. If we want to query minimal bottleneck path between any two vertices, we can use dfs to calculate maxcost array just as we do in Second-best Minimal Spanning Tree. Then we can answer each query in $O(1)$, and the total complexity is $O(n^2)$.
  \subsection{Minimal Directed Spanning Tree}
  \begin{lstlisting}[language=C++]
const int INF = 1000000000;
const int maxn = 100 + 10;

struct MDST {
  int n;
  int w[maxn][maxn]; 
  int vis[maxn];
  int ans;         
  int removed[maxn];
  int cid[maxn];
  int pre[maxn];
  int iw[maxn];
  int max_cid;

  void init(int n) {
    this->n = n;
    for(int i = 0; i < n; i++)
      for(int j = 0; j < n; j++) w[i][j] = INF;
  }

  void AddEdge(int u, int v, int cost) {
    w[u][v] = min(w[u][v], cost);
  }

  int dfs(int s) {
    vis[s] = 1;
    int ans = 1;
    for(int i = 0; i < n; i++)
      if(!vis[i] && w[s][i] < INF) ans += dfs(i);
    return ans;
  }

  bool cycle(int u) {
    max_cid++;
    int v = u;
    while(cid[v] != max_cid) { cid[v] = max_cid; v = pre[v]; }
    return v == u;
  }

  void update(int u) {
    iw[u] = INF;
    for(int i = 0; i < n; i++)
      if(!removed[i] && w[i][u] < iw[u]) {
        iw[u] = w[i][u];
        pre[u] = i;
      }
  }

  bool solve(int s) {    
    memset(vis, 0, sizeof(vis));
    if(dfs(s) != n) return false;

    memset(removed, 0, sizeof(removed));
    memset(cid, 0, sizeof(cid));
    for(int u = 0; u < n; u++) update(u);
    pre[s] = s; iw[s] = 0;
    ans = max_cid = 0;
    for(;;) {
      bool have_cycle = false;
      for(int u = 0; u < n; u++) if(u != s && !removed[u] && cycle(u)){
        have_cycle = true;
        int v = u;
        do {
          if(v != u) removed[v] = 1;
          ans += iw[v];
          for(int i = 0; i < n; i++) if(cid[i] != cid[u] && !removed[i]) {
            if(w[i][v] < INF) w[i][u] = min(w[i][u], w[i][v]-iw[v]);
            w[u][i] = min(w[u][i], w[v][i]);
            if(pre[i] == v) pre[i] = u;
          }
          v = pre[v];
        } while(v != u);        
        update(u);
        break;
      }
      if(!have_cycle) break;
    }
    for(int i = 0; i < n; i++)
      if(!removed[i]) ans += iw[i];
    return true;
  }
};
  \end{lstlisting}



 \section{Bipartite Graph Matching}
  \subsection{Augmenting Path Algorithm}
  \textbf{Test:POJ 3020}
  \begin{lstlisting}[language=C++]
int n,m; //size of bipartite sets
int G[maxn][maxn]; //adjacency matrix
int linker[maxn],vis[maxn];
int dfs(int u)
{
	for(int v=0;v<m;v++)
		if(G[u][v] && !vis[v])
		{
			vis[v]=1;
			if(linker[v]==-1 || dfs(linker[v]))
			{
				linker[v]=u;
				return 1;
			}
		}
	return 0;
}
int hungary()
{
	int ans=0;
	memset(linker,-1,sizeof(linker));
	for(int u=0;u<n;u++)
	{
		memset(vis,0,sizeof(vis));
		if(dfs(u)) ans++;
	}
	return ans;
}
  \end{lstlisting}
  \subsection{Kuhn-Munkers Algorithm}
  \textbf{Test:UVaLive 4043}
  \begin{lstlisting}[language=C++]
int W[maxn][maxn],n;
int Lx[maxn],Ly[maxn];	//label on X and Y
int linker[maxn],slack[maxn];
int S[maxn],T[maxn];	//mark whether visited
int dfs(int u)
{
	S[u]=1;
	for(int v=0;v<n;v++) 
	{
		if(T[v]) continue;
		int tmp=Lx[u]+Ly[v]-W[u][v];
		if(tmp==0)
		{
			T[v]=1;
			if(linker[v]==-1 || dfs(linker[v]))
			{
				linker[v]=u;
				return 1;
			}
		}
		else if(slack[v]>tmp)
			slack[v]=tmp;
	}
	return 0;
}
void update()
{
	int a=inf;
	for(int i=0;i<n;i++)
		if(!T[i] && a-slack[i]>0)
			a=slack[i];
	for(int i=0;i<n;i++)
	{
		if(S[i]) Lx[i]-=a;
		if(T[i]) Ly[i]+=a;
		else slack[i]-=a;
	}
}
void KM()
{
	for(int i=0;i<n;i++)
	{
		linker[i]=-1;
		Lx[i]=Ly[i]=0;
		for(int j=0;j<n;j++)
			Lx[i]=max(Lx[i],W[i][j]);
	}
	for(int i=0;i<n;i++)
	{
		for(int j=0;j<n;j++) slack[j]=inf;
		while(1)
		{
			memset(S,0,sizeof(S));
			memset(T,0,sizeof(T));
			if(dfs(i)) break;
			else update();
		}
	}
}
  \end{lstlisting}



 \section{Network Flows}





 \section{Depth First Search}
  \subsection{Calculate DFS Sequence}
  \textbf{Test:CF 383C}
  \begin{lstlisting}[language=C++]
//tree is stored in adjacency lists
//dfs_clock is initialed as 0,vis array is initialed as 0
//dfn is the dfs sequence,ldfs is the time the vertex enters and rdfs the time leaves 
int ldfs[maxn],rdfs[maxn],dfn[maxn],vis[maxn];
int dfs_clock;
void dfs(int u)
{
    vis[u]=1;
    ldfs[u]=++dfs_clock;
    dfn[dfs_clock]=u;
    for(int i=head[u];i!=-1;i=nxt[i])
    {
        int v=pnt[i];
        if(!vis[u]) dfs(v);
    }
    rdfs[u]=++dfs_clock;
    dfn[dfs_clock]=u;
}
  \end{lstlisting}

  \subsection{Multiplication LCA Algorithm(online)}
  \textbf{Test:POJ 1330,3728}
  \begin{lstlisting}[language=C++]
const int pow=20;
int d[maxn],p[maxn][pow];
int head[maxn],nxt[maxn],pnt[maxn];

void dfs(int u,int fa)                  //get d and p array
{
	d[u]=d[fa]+1;
	p[u][0]=fa;
	for(int i=1;i<pow;i++) p[u][i]=p[p[u][i-1]][i-1];
	for(int i=head[u];i!=-1;i=nxt[i])
	{
		int v=pnt[i];
		dfs(v,u);
	}
}
int lca(int a,int b)                    //get the lca of a and b in O(logn)
{
	if(d[a]>d[b]) a^=b,b^=a,a^=b;
	if(d[a]<d[b])
	{
		int del=d[b]-d[a];
		for(int i=0;i<pow;i++) if(del&(1<<i)) b=p[b][i];
	}
	if(a!=b)
	{
		for(int i=pow-1;i>=0;i--)
			if(p[a][i]!=p[b][i])
				a=p[a][i],b=p[b][i];
		a=p[a][0],b=p[b][0];
	}
	return a;
}
  \end{lstlisting}

  \subsection{Targan's Off-line LCA Algorithm}
  \textbf{Test:POJ 1330,1470}
  \begin{lstlisting}[language=C++]
vector<int> g[maxn],q[maxn];//adjacency lists and query lists
int f[maxn];
int a[maxn];//record the current lca of the vertex
int vis[maxn];
//disjoint set
int find(int x) {return x==f[x]?x:f[x]=find(f[x]);}
void uni(int x,int y)
{
	int a=find(x);
	int b=find(y);
	if(x!=y) f[a]=b;
}
//use:lca(root);
//usually insert (u,v) and (v,u) into q
void lca(int u)
{
	a[u]=u;
	for(int i=0;i<g[u].size();i++)
	{
		lca(g[u][i]);
		uni(u,g[u][i]);
		a[find(u)]=u;
	}
	vis[u]=1;
	for(int i=0;i<q[u].size();i++)
	{
		if(vis[q[u][i]]) 
		{
		    int LCA=a[find(q[u][i])];   //answer the query
		    operation();                //operate corresponding operations
		}
	}
}
  \end{lstlisting}









\endinput